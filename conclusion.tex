\section{Conclusion}
\label{sec:conclusion}

In this work, we propose a novel approach for global stiffness struture optimization and present a saddle point algorithm to solve the optimization problem efficiently.
%
When given a certain amount of material for printing an object in 3D, our approach can generate a global stiffness structure with minimum deformation under all possible force distributions.
%
Our approach also provides a solution for formulating adaptive hollowing and interior supportive structures in a unifed form, while optimizing them simultaneously.
%
A number of experimental results have shown the validity and the rationality of our solution,
and have proved our proposed approach to be much more applicable than previous methods.
%


\noindent\textbf{Limitations and future work}
Our research opens several future studies in the direction of structural optimization.

In this paper, the optimization's objective is to minimize the possible deformation of objects.
However, the maximum stress distribution inside an object is of more interest in some applications, since it tells us where a crack is likely to happen. For such a need, we should introduce new objective functions. A simple idea is to consider the following optimization problem.
$$
\min_{(V,r)}\max_{f} \frac{\int_{\Omega} |\sigma|^2 dx}{\int_\Omega f^2dx}
$$
where $\sigma$ denotes the stress of the body under a force $f$.


%\xuefeng{My revision ends here.}


%From beam theory we know that lightweight frame structures under small load can be considered as linear elastic structure.
%Nonlinear elasticity has to be investigated for cases of frame under large deformation.
%In nonlinear elasticity, the displacements of nodes are not linearly related to the loads, so the linear stiffness equation is no longer valid.
%%
%
%%There are two dual problems in the structural optimization:
%%maximizing the strength of designed structure with given material volume constraint;
%%and minimizing the volume of material usage to meet a given strength/deformation constraint on the structure.
%
%
%In our current work, we consider the structural optimization problem of maximizing the global stiffness of designed structure with a given amount of material constraint.
%%
%Here the maximum global stiffness structure is defined as that provides minimal deformation under various force distribution.
%If we consider the maximum strength structure that produces minimal stress under various force distribution, the structural optimization problem appears to be solvable but would be challenging.
%%
%Second, .... some topic is an intriguing direction for future research.

