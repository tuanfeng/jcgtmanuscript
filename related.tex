\section{Related Work}
\label{sec:related}


%\hl{Technologies around 3D printing have been popular since recent years. The capability of generating tangible solid objects with fine surface details and complex interior structures from their digital representation inspires significant efforts in relative areas. In practice, 3D printers are a powerful and affordable commercial solution to popularize the self-prototyping of custom-designed sphtsical objects and complex inner structure design, which opened new horizons in 3D shape design technologies for fabrication.} \zhouwang{move this part to introduction?}


Significant efforts have been made recently on fabrication-aware 3D shape design which establishes physically-fabricated prototypes and manufactured objects using 3D printing. Previous work focused on the desire of certain physical properties, such as deformation behavior~\cite{Bickel:2010}, fabricatable via comercial 3D printer's limited working volume~\cite{luo2012chopper}, physical realization~\cite{stava:2012}, stability over gravity on certain orientation~\cite{prevost:2013, bacher2014spin}. These work fully developed the benefit of 3D printing over traditional CNC fabrication techniques because both fine surface details and complex interior structure are able to be fabricated.

Over the recent years it has been more and more recognized that 3D printing techniques, including FDM/SLS/SLA, etc., are hindered for both research and commercial purposes by such a high material cost and a rather low fabrication speed. The core problem here is to reduce the designed volume of the object. Inspired by lightweight structure observed from nature, \cite{wang:2013, Lu:2014} adopted lightweight structures to fill the object rather than solid interior. These work optimize the stiffness/strength-to-weight ratio under a given external force while the original object surface is preserved. However, the results of these load-based methods fall short in real world cases as load distribution applied by user is always different from the ideal preset case. With a novel formulation, our proposed unknown-load-based method perfectly solves this problem by optimizing and generating a global stiffness design with a uniform framework. Without a given load distribution, \cite{umetani2013cross} presents a structural analysis technique that slice the object into cross-sections and compute stress based on bending momentum equilibrium. For a similiar purpose, a finite element based structural analysis method is presented in~\cite{zhou:2013}. Based on experimental observation, this work uses an Eigenmode formulation to detect the weakest area of an object. In our paper, this observation is proved mathmatically from our global stiffness formulation. In computational structure area, structure analysis and optimization under unknown load distribution have been discussed during the past few years, such as \cite{cherkaev2004principal,cherkaev1998stable,takezawa2011topology}. These work, although sharing the same target, test different kinds of object functions without a proof of rationality. These work also limited the problem into 2D domain and focus on only geometry rather than take all variables (size, geometry, topology) into consideration, which is far from a complete solution of the cost-effective fabrication challenge.

In order to reduce the volume of the object, lightweight structure is adapted for supportive interior. The design and optimization of lightweight structures have been extensively explored in tissue engineering and computer-aided design. 
Smith et al.~\cite{Smith:2002} focus on optimizing the design of truss structures where beams connected by pin joints are rotation-free and difficult to preserve the geometrical shape of the object.
Using lightweight structures for improving the strength and the stiffness of objects has also been studied in the field of rapid manufacturing~\cite{wang:2005}, where the particle swarm optimization or generic algorithm were selected to search for design solutions. Detailed reviews on various aspects of structural optimization can be found in~\cite{bendsoe:2003}. Due to the differences in the types of objectives and constraints, the approaches there in are not suitable for our purpose of 3D printing.

We use the frame structure in the lightweight design for its fabrication-friendly. Each beam in the frame structure is considered as the basic element which is different from classical structural topology optimization methods. Structural topology optimization is employed mainly to specify the optimum number and location of holes in the configuration of the designed structure.
Element-based methods for structural topology optimization decompose the volume of the input object into finite-element-like tetrahedrons, or a grid-like structure for analysis. The \emph{Optimality Criteria} (OC)~\cite{rozvany:1989} methods are proved to be among the most effective element-based methods for solving topology optimization problems. 
A recent review of this area is offered by~\cite{rozvany:2009}.
It is easy to extend our framework into element-based computation, however, due to the linear elastic property of frame structure, taking a single beam as basic element is much more efficient.
